\section{Premise}

The premise of this treatise is that it must be possible to create
a binary encoding format specification such that any random input 
deterministically produces a valid output.
The use of this specification must allow for complex data structures
in the output (for example, as offered by ASN.1, XML or JSON),
and even useful ones (in other words, this is not just a joke).

\section{Problem Statement}

Binary encodings are often chosen because they are (supposedly) dense.
However, when applied, they often contain at least some 'air'.
The problem with binary encodings is that this 'air' can also contain
encoding mistakes. For example:

\begin{itemize}
\item When using TLV, they may contain types that aren't defined.
\item When using TLV, they may contain lengths that are illegal.
  Either because they are zero, which the type does not support,
  or because they exceed the length of the message size, or indeed
  the size of the element that contains them.
\item For example, in DER encoding it is possible to defines lengths
  that could not possibly exist in our universe.
\item For example, in Unicode, there exist illegal encodings, when the
  amount of bytes part of the definition of a character, aren't
  available, or when the resultant character value isn't specified.
\end{itemize}

Textual formats can convey complex data structures, but are also
(very) error prone. For example:

\begin{itemize}
\item String escaping may lead to illegal escape sequences or breaking
  of string enclosing logic.
\item Closing strings and brackets may be ommitted or over-supplied.
\item Closing tags - same thing.
\item Illegal bytes in 7-bits clean formats.
\end{itemize}
etc.

All of these formats are problematic because we have to deal with
errors during parsing.

Funnily, some formats are binary yet don't suffer from these at all.
Examples are raw audio, image and video encodings. So long as the
encoder and decoder have previously agreed on resolution,
any further bit in a video stream is simply the next bit used to
create the current pixel, on the current scanline, in the current frame.

Granted, in audio and video we can under- or overflow the input buffer of the
decoder in time. And in images we can have an incomplete buffer.
But what input we give it, cannot be 'illegal'. If we feed
it random bits, the screen will show static, but it will be 'legal static'.

\section{Design Principles}
 
\begin{itemize}
\item Length can never be used, as it may under- or overflow the message
  or message segment size.
\item Our atom will be a bit, not a byte, as we cannot have any 'air'
  in our encoding. Implementations will be bit-shifting a lot, I think.
\item Complexity of the output must be 'like ASN.1, XML or JSON'.
  That is to say:
  \begin{itemize}
  \item Encodes all eight-bit bytes and all lengths of those bytes.
  \item Is complex in that it provides at least for strings, integers,
    floating points, 'hashtable' name-value lists, and arrays.
  \end{itemize}
\item We don't care about being truly efficient. The fact that we're
  binarily encoding should bring us enough of that. The main goal stands:
  any decoder input must produce an output.
\item We don't care about deterministic reciprocity between encoding
  and decoding: if the input to the decoder produces
  a complex data structure, then the complex data structure, when fed
  to the encoder, doesn't need to produce a copy of the input.
  The fact that, for example, values in name/value lists are implicit,
  makes this a necessity.
\item There exists a 'null' type (in fact, there are two).
  Which can be given explicitly, or
  we use it to fill in all sorts of blanks.
\item Everything is implicitly a list. When an implicit list contains
  one element, it is a scalar. If you want to define a list with
  one element, you must explicitly define a list.
\item We use 'TV' encoding (leaving out the 'L'ength).
  When we are looking for a type (because the previous tuple decoding
  has finished) we simply assume that the following bits are part of the
  new tuple in the implicit list that we're processing.
\end{itemize}

\section{Specification}

Explicitly written from the standpoint of the decoder.
The encoder will use this specifiction simply in reverse.

\begin{itemize}
\item There are the following types (eight, so that they fit exactly
      into three bits):
  \begin{itemize}
  \item NULL
  \item Explicit NULL
  \item BOOLEAN
  \item INTEGER (64 bits signed)
  \item FLOAT (of system double size)
  \item STRING
  \item ARRAY
  \item HASHTABLE
  \end{itemize}

\item The decoder starts assuming it has to fill a list.
  If, in the end, this list will contain exactly one element, it will
  return the element, not the list. In all other cases (empty list,
  list with more than one element), it will return the list.

\item The decoder pops a type triplet and switches according to its value:
  \begin{itemize}
  \item If it is a null, it will move to pop the next tuple.
  \item If it is a boolean, it will pop the next bit to determine its value
        and move on to the next tuple.
  \item If it is an int or a float, it will pop the next 64 bits,
        cast them to the machine representation, and move on to the next tuple.
  \item If it is a string, it will go into a loop determining, per byte,
        whether or not to read it. It does this by popping a bit and if
        it is one, it will pop the byte. If it is zero, it assumes the
        string is finished and move on to the next tuple.
  \item If it is a list, it does the same 'continuity bit popping' for
        each list element, and each element recurses into the list-tuple
        decoding.
  \item If it is a hashtable, it pops a string (without type triplet popping)
        as key, and an element (as in the list), and does this in the same
        'continuity bit popping' kind of way as with strings or list elements.
  \end{itemize}

\item Should an input end with one to seven zero bits, these should, per the
  specification above, be rendered as a list of one or two NULL types
  (one null D-bit and three null type-bits, which can repeat a maximum of
  two times in seven bits), these will only not be discarded when they
  are explicit NULL types.

  Note that \textit{eight} null bits are explicit (in that they encode
  an array of two null values).
\end{itemize}

\section{Thoughts on Specific Forms of Sub-Encoding}

\subsection{Bigger type space encoding}

Currently, the amount of types has been limited to eight. This happened
to satisfy the requirements of encoding JSON however, it may be that you're
having a more extensive need for type encoding. In that case one could
extend the type identifyer sequence with one bit, giving you sixteen
possible types.

If then, however, you're not in need of all sixteen (or 32, or 64, or ...)
types, you'll be forced to apply a modular approach. Your type then
becomes the modulo of type space available, divided by the amount of
types needed. This will also make the encoder/decoder stages less
deterministically linked (since you can have, in some cases, more than
one discriminant for a type).

\subsection{String Encoding}

Currently, the implementation pulls a single bit to see if it would need
to read another single byte. This creates a string encoding overhead of
12.5\%, which may be seen as extensive.

It is relatively easy however, to create a string encoding scheme that
is more frugal. For example: pull three bits to determine how many of the
following 0-7 bytes should be read. Overhead then quickly drops to low
single digit percentages (1.9\% in this case).

\subsection{Integer Encoding}

Currently, integers are encoded as eight byte integers. This:

\begin{itemize}
\item May not sit well with 32 bit platforms.
\item May not satisfy those who require bigints (for example, for use in
      cryptographic applications).
\end{itemize}

... section to be written

\subsection{Floating Point Number Encoding}

... section to be written

\section{Examples}

\textit{
Binary data representations can be given in two formats, bit-wise and byte-wise.
Bit-wise data is preceeded, per section, by a 'B:',
while byte-wise data is preceeded, per section, by a 'H:' (and is denoted in
hexadecimal tuplets).
}

The section below needs to be rewritten. Ignore.

The following input:

\begin{changemargin}{-60mm}{0mm}
\begin{myquote}
\begin{verbatim}
\end{verbatim}
\end{myquote}
\end{changemargin}

Produces the following JSON:

\begin{changemargin}{-60mm}{0mm}
\begin{myquote}
\begin{verbatim}
\end{verbatim}
\end{myquote}
\end{changemargin}

The following input:

\begin{changemargin}{-60mm}{0mm}
\begin{myquote}
\begin{verbatim}
\end{verbatim}
\end{myquote}
\end{changemargin}

Produces the following JSON:

\begin{changemargin}{-60mm}{0mm}
\begin{myquote}
\begin{verbatim}
\end{verbatim}
\end{myquote}
\end{changemargin}


